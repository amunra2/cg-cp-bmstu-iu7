\chapter*{Введение}
\addcontentsline{toc}{chapter}{Введение}

В современном мире компьютерная графика вышла на совершенно новый уровень, что связано с её высокой востребованностью в области игр и фильмов \cite{realistic-water}. Методы отрисовки все время развиваются, появлятся новые способы. Главным требованием является, в первую очередь, является реалисточнось изображения. Множество исследований физических явлений происходит постоянно, чтобы с максимальной точностью смоделировать тот или иной объект. Но чем выше точность, тем выше сложность разрабатываемых алгоритмов, что часто приводит к высоким затратам по времени и памяти.

Одной из самых сложных тем при моделировании являются жидкости \cite{water-matters}. Имеется огромная потребность в качественной и эффективной отрисовке океанов, рек, ручейков и даже луж. В данном курсовом проекте речь пойдет о моделировании водопада, с учетом аэорозольных облаков, которые возникают у подножия самого водопада при падении воды.

Цель работы - создать качественную симуляцию водопада с использованием современных методов и технологий

\begin{itemize}
	\item проанализировать методы и алгоритмы, моделирующие водопады; 
	\item определить алгоритм, который наиболее эффективно справляется с поставленной задачей; 
	\item реализовать алгоритм;
\end{itemize}

