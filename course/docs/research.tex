\chapter{Исследовательская часть}

В данном разделе будут приведены примеры работы программа, а также поставлен эксперимент по сравнению производительности программы.


\section{Демонстрация работы программы}

На рисунках \ref{img:example1}--\ref{img:example3} представлены результаты работы программы. При этом на рисунке \ref{img:interface_wf} представлено окно управления водопадом, а на рисунке \ref{img:interface_cam} -- окно управления камерой.


% \clearpage

\imgHeight{90mm}{example1}{Пример работы программы (вид 1)}

\imgHeight{90mm}{example2}{Пример работы программы (вид 2)}

\imgHeight{90mm}{example3}{Пример работы программы (вид 3)}

\imgHeight{70mm}{interface_wf}{Окно управления водопадом}

\imgHeight{70mm}{interface_cam}{Окно управления камерой}


\clearpage


\section{Постановка эксперимента}

\subsection{Цель эксперимента}

Целью эксперимента является проведение тестирования производительности при создании сцен различной нагруженности. Нагрузка будет меняться в зависимости от количества частиц, из которых состоит водопад.

Оцениваться производительность будет мерой количества кадров в секунду (Frames Per Second, FPS, к/с), которое получается при работе приложения при данной загруженности.


\subsection{Технические характеристики}

Технические характеристики устройства, на котором выполнялось тестирование представлены далее:

\begin{itemize}
    \item операционная система: Ubuntu 20.04.3 \cite{ubuntu} Linux \cite{linux} x86\_64;
    \item память: 8 GiB;
    \item процессор: Intel® Core™ i5-7300HQ CPU @ 2.50GHz \cite{intel};
    \item видеокарта: NVIDIA® GeForce® GTX 1050Ti with 4 GB GDDR5 Dedicated VRAM \cite{gtx1050}.
\end{itemize}

При тестировании ноутбук был включен в сеть электропитания. Во время тестирования ноутбук был нагружен только встроенными приложениями окружения, а также системой тестирования.


\subsection{Результаты эксперимента}

Результаты эксперимента приведены в таблице \ref{tbl:fps_compare}. Также на рисунке \ref{img:fps_compare} представлен график изменения FPS в зависимости от количества частиц.

\begin{center}
    \captionsetup{justification=raggedright,singlelinecheck=off}
    \begin{longtable}[c]{|c|c|}
    \caption{Зависимость производительности от количества частиц\label{tbl:fps_compare}}\\ \hline
        \textbf{Частиц, штук} & \textbf{Производительность, к/c} \\ \hline
        500   & 210 \\ \hline
        1000  & 120 \\ \hline
        2000  & 80 \\ \hline
        3000  & 50 \\ \hline
        4000  & 34 \\ \hline
        5000  & 27 \\ \hline
        6000  & 21 \\ \hline
        7000  & 18 \\ \hline
        8000  & 16 \\ \hline
        9000  & 14 \\ \hline
        10000 & 12 \\ \hline
        11000 & 11 \\ \hline
        12000 & 10 \\ \hline
        13000 & 9 \\ \hline
        14000 & 8 \\ \hline
        15000 & 7 \\ \hline
\end{longtable}
\end{center}

\imgHeight{120mm}{fps_compare}{Зависимость к/с от количества частиц в водопаде}

\clearpage

\subsection*{Вывод эксперимента}

Как видно из результатов, количество кадров в секунду уменьшается экспоненциально при линейном увеличении количества частиц в водопаде. Рендер большого количества частиц, несмотря на хорошее программное обеспечение компьютера, является трудной задачей: при 1000 частиц программа выдает 120 FPS, в то время как при уже при 10000 тысячах частиц получается 12 FPS. 

Для комфортной работы человеку необходимо 24 FPS \cite{fps}, что получается при 5000 тысячах частиц. При этом частиц достаточно, чтобы они вместе были похожи на водопад.


\section*{Вывод}

В данном разделе были рассмотрены примеры работы программного обеспечения, а также выяснено в результате эксперимента, что производительность программы (в FPS) падает по экспоненциальному закону при линейном увеличении количества частиц.

