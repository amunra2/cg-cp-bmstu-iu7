\chapter{Конструкторская часть}

В данном разделе будут представлены требования к программному обеспечению, а также схемы алгоритмов, выбранных для решения задачи.


\section{Требования к программному обеспечению}

Программа должна предоставлять доступ к функционалу:

\begin{itemize}
    \item изменение параметров модели водопада в активном режиме: высота водопада, угол падения воды, скорость водяного потока;
    \item изменение параметров частиц, из которых состоит водопад, в активном режиме: количество частиц, размер частиц;
    \item включение и выключение работы модели водопада;
    \item вращение, перемещение и масштабирование модели.
\end{itemize}

Требования, которые предъявляются к программе:

\begin{itemize}
    \item время отклика программы должно быть менее 1 секунды для корректной работы в интерактивном режиме;
    \item программа должна корректно реагировать на любое действие пользователя.
\end{itemize}


\section{Разработка алгоритмов}

В данном разделе будут представлены схемы реализации выбранных алгоритмов.

\subsection{Система частиц для реализации водопада}

На рисунке \ref{img:waterfall_scheme.png} показана схема алгоритма реализации движения частицы и возможные этапы ее превращения в пар и брызг.

\imgHeight{200mm}{waterfall_scheme.png}{Схема алгоритма движения частицы водопада}


\subsection{Отрисовка изображения}

На рисунке \ref{img:draw_dynamic.png} показана схема алгоритма отрисовки части сцены, отвечающей за водопад, а на рисунке \ref{img:draw_solid.png} -- схема алгоритма для отрисовки остальных объектов сцены (скалы и водного полотна).

\imgHeight{200mm}{draw_dynamic.png}{Схема алгоритма отрисовки водопада}
\imgHeight{200mm}{draw_solid.png}{Схема алгоритма отрисовки статичного объекта (скалы)}

\clearpage


\section{Описание структуры программы}

На рисунке \ref{img:classes_scheme.png} показана структура реализуемых классов.

\imgHeight{100mm}{classes_scheme.png}{Схема классов программы}

\clearpage

Описание реализуемых классов:

\begin{itemize}
    \item \textbf{Camera} -- класс для работы с камерой. Хранит позицию камеры, угол ее наклона, скорость перемещения для динамической смены положения камеры;
    \item \textbf{Shader} -- класс, который подключает шейдеры к приложению. Хранит данные шейдеров для передачи цвета объектов;
    \item \textbf{Object} -- класс, который владеет информацией о всех объектах сцены;
    \item \textbf{winGL} -- класс для отрисовки объектов сцены. Хранит массив частиц для водопада, хранит массив цветов для водопада, а также данные для отрисовки статичных объектов;
    \item \textbf{Particle} -- класс, который описывает одну частицу водопада. Хранит ее положение, направление движения, цвет, скорость, время жизни и максимальное время жизни. 
\end{itemize}


\section{Используемые типы и структуры данных}

При реализации программного обеспечения были использованы следующие типы и структуры данных:

\begin{itemize}
    \item параметры водопада: высота, ширина, скорость, размер частиц -- числа типа $float$;
    \item параметры водопада: количество частиц -- число типа $int$
    \item точка -- массив координат по осям X, Y, Z;
    \item объект -- массив точек с координатами вершин, также массив связей между номерами вершин;
    \item водопад -- массив частиц (объектов класса $Particle$).
\end{itemize}


\section*{Вывод}

В данном разделе были рассмотрены требования, которые выдвигаются программному продукту, схемы алгоритмов, а также типы и структуры данных, которые были использованы при реализации ПО. 
