\chapter{Технологическая часть}

В данном разделе будут рассмотрены средства разработки программного обеспечения, детали реализации, а также диаграмма классов.


\section{Средства реализации}

При написании программного продукта был выбран язык \textit{Python} \cite{python-lang}. Это обусловлено следующими факторами:

\begin{itemize}
    \item объектно-ориентированный язык, что позволяет использовать структуру классов;
    \item имеются необходимые библиотеки для реализации поставленной задачи;
    \item существует много учебной литературы.
\end{itemize}

В качестве разработки интерфейса был выбран \textit{Qt Designer} \cite{qt-designer}. Он позволяет создать качественный интерфейс, так как имеет встроенный редактор выводимого окна.

В качестве среды разработки был выбран \textit{Visual Studio Code} \cite{vs-code}. Данное приложение имеет большое количество плагинов для работы с кодом.


\section{Реализация алгоритмов}

В листинге \ref{lst:particle_move} представлен алгоритм перемещения частицы водопада за один кадр.

\begin{center}
    \captionsetup{justification=raggedright,singlelinecheck=off}
    \begin{lstlisting}[label=lst:particle_move,caption=Алгоритм перемещения частицы водопада за один кадр]
def moveWaterfallParticle(self):
    if (self.pos[1] <= WATER_LINE):
        self.speed = BOUNCE_COEF * self.speed
        self.direction[2] *= -1
        self.vaporized = True

    oldSpeed = self.speed
    oldDirection = deepcopy(self.direction)

    self.direction[0] += self.gravityDirection[0]
    self.direction[1] += self.gravityDirection[1]
    self.direction[2] += self.gravityDirection[2]

    if (self.vaporized):
        self.speed -= self.acceleration

        self.pos[0] = self.pos[0] + oldSpeed * oldDirection[0] + (self.speed * self.direction[0]) / 2 
        self.pos[1] = self.pos[1] - (oldSpeed * oldDirection[1] + (self.speed * self.direction[1]) / 2)
        self.pos[2] = self.pos[2] - (oldSpeed * oldDirection[2] + (self.speed * self.direction[2]) / 2)
    else:
        self.speed += self.acceleration

        self.pos[0] = self.pos[0] + oldSpeed * oldDirection[0] + (self.speed * self.direction[0]) / 2
        self.pos[1] = self.pos[1] + oldSpeed * oldDirection[1] + (self.speed * self.direction[1]) / 2
        self.pos[2] = self.pos[2] + oldSpeed * oldDirection[2] + (self.speed * self.direction[2]) / 2

    self.age += 1

    if (self.age + 100 > self.maxAge):
        self.speed *= 0.9
        self.direction[2] *= 1.014
        self.color = glm.vec4(1, 1, 1, 1)

    return self
\end{lstlisting}
\end{center}


\section*{Вывод}

В данном разделе были рассмотрены средства, с помощью которых было реализовано ПО, описана структура классов проекта, а также представлен листинг алгоритма перемещения частицы водопада за кадр.
