\chapter{Аналитическая часть}

\section{Водопад как природное явление}

Водопад - падение воды в реке с уступа, который пересекает речное русло. Не всякий поток воды можно накзвать водопадом. Для этого должны соблюдаться несколько важных условий - высота более 1 метра и уклон свыше 45 градусов. 

\img{60mm}{waterfall_model.jpg}{Модель водопада}

Чаще всего водопады возникают при перепаде высот, которые были там еще до появления реки. 
Также водопады появляются по причине эрозии - водяной поток размывает мягкие осадочные породы, тем самым происходит обрушение вышележащих слоев породы, вследствие чего на границе твердых и мягких пород образуется вертикальный уступ, с которого падает вода.

\img{60mm}{waterfall_consists.jpg}{Составные части водопада}

При моделировании водопада чаще всего обращаюат внимание лишь на поток воды, который падает вниз с обрыва. Но для качественной и реалистичной картинки этого не достаточно. При построении модели водопада нужно учитывать три важные его составляющие - сам поток воды (water stream), брызги (splashing spray), которые от него исходят, а также аэрозольное облако (spray cloud), которое образуется при ударе потока об воду.

В процессе моделирования каждая часть будет рассмотрена отдельно, чтобы в конце совместить результаты и получить наиболее реалистичную картинку.

\section{Методы визуализации водопадов}

Водопад представляет собой водяной поток с частицами брызг. За все время моделирования текучей воды было разработано несколько основных методов, конкретнее о которых речь и пойдет дальше.

\subsection{Метод, основанный на уравнении Навье-Стокса}

\textit{Уравнение Навье-Стокса} \cite{navier-stokes} является системой дифференциальных уравнений в частных производных, которое описывает движение вязких ньютоновских жидкостей, которое до сих пор не имеет решения в общем виде. Оно часто используется для математического моделирования сложных моделей природных явлений. При разработке алгоритмов используются частные решения.

Методы, которые строятся на основе этого уравнения, являются довольно трудно реалиуземы за счет большого количества сложной математики, с которой приходится работать, а также множества вычислений, которые приходится делать компьютеру при обработке всех формул, что является крайне неэффективным. По этим причинам методы на основе уравнения Навье-Стокса рассматриваться подробно не будут. \cite{navier-stokes-cg}

\subsection{Метод, основанный на частицах}

Метод частиц заключается в использовании трилинейной интерполяции для каждой частицы, чтобы определить ее скорость движения. И каждая частица перемещается в соответствии с определенным инерциальным физическим уравнением. Данный подход имеет довольно небольшие затраты по компюьтерным ресурсам, а также большое количество частиц позволяет достичь максимальной точности изображения. Главным минусом является невозможность точно определить границы, в которых должен находиться водяной поток

Наиболее распространённые методы с частицами:
\begin{itemize}
    \item метод на основе понятия о полу-Лагранже, который объединяется с новым подходом расчета жидкости вокруг объектов. Это повзволяет эффективно решать уравнения движения жидкости, сохраняя при этом достаточно деталей, чтобы получить реалистичное изображение. Высококачественная поверхность получается из результирующего поля скоростей с использованием новой адаптивной техники для создания неявной поверхности  \cite{semi-lagrangian-particle};
    \item метод, использующий диаграммы Вороного. Ключевым компонентом этого алгоритма является аппроксимация геометрии пены путем обработки частиц пузырьков как участков взвешенной диаграммы Вороного. Информация о связности, предоставляемая диаграммой Вороного, позволяет нам точно моделировать различные эффекты взаимодействия между пузырьками.Используя ячейки Вороного и веса, мы также можем явно решить проблему потери объема при моделировании пены, что является общей проблемой во многих подходах. Может быть встроен в симуляторы жидкости, в частности, водопада \cite{voronov-diagram-particle}.
\end{itemize}

\subsection{Сеточные методы}

Сетки часто используются при моделировнии жидкостей. В свою очередь, сама сетка - регулярная кубическая, в каждой ячейке которой содержится информация о точке поверхности, которая задает саму картину. Данный метод при большом количестве ячеек вызывает большое потребление памяти, хотя чем больше ячеек, тем выше качество получаемого изображения. Но также этот метод имеет и  свои преимущества. Во-первых, визуализировать результат симуляции проще. Во-вторых, точно известны границы желаемой области, что приводит к возможности встроить данный водяной поток в нужное место.

Наиболее распространённые методы с сеткой:
\begin{itemize}
    \item метод моделирования пузырька воздуха в воде. Этот метод представляет собой физически обоснованный метод расчета размеров пузырьков воздуха, основанный на скорости замерзания и давлении. В отличие от большинства методов, этот метод может представлять множество мелких пузырьков, которые не могут быть представлены с помощью решений имитационных сеток \cite{bubble-grid};

    \item метод, использующий уравнение Эйлера. Благодаря новому подходу можно моделировать крупномасштабные трехмерные жидкости. Здесь используется гибридное представление сетки, состоящее из правильных кубических ячеек поверх слоя высоких ячеек. Также благодря оптимизации в представлении сетки была предлоежена реализация ГПУ решателя для жидкости \cite{eiler-equation-grid}.
\end{itemize}


\subsection{Комбинированный метод}
Поскольу каждый метод по отдельности - на частицах и на сетке - имеет свои недостатки, то было предложено объединить эти два метода в один. Совместное их использование помогает одновременно с потоком воды моделировать брызги, пузырьки. Также подобные методы имеют высокую эффективность за счет несложных математических вычислений. Комбинированный метод частиц и сетки и ложится в основу алгоритма моделирования водопада, который будет рассматриваться в дальнейшем

Наиболее распространённые комбинированные методы:
\begin{itemize} 
    \item метод, при котором моделируются пузырьки воздуха и пена вместе. Обе фазы рассматриваюися отдельно. Также в этом методе применяется такие подоходы, как использование силы сопротивления (для того, чтобы совместить оба этапа и при этом сильно не увеличить вычислительные нагрузки), функция насыщения (завсисит от объема). При достижении поверхности пузырьки преобразуются в пену, а потом удаляются за указанное пользователем время \cite{sph-combo};
    \item метод, который напрямую использует подходы на основе сеток и частиц. Области ждикости, которые не могут быть представлены полем полем высоты (например, разбивающиеся волны и брызги из-за ударов воды), то они представляются частицами и превращаются в брызги и пену. Сами частицы рассматриваются как просте невзаимодействующие точечные массы, которые обмениваются массой и моментом с жикодстью поля высоты  \cite{real-time-combo}.
\end{itemize}


\subsection*{Вывод}
В качестве приоритетного выбирается комбинированный метод визуализации водопада. Он имеет преимущества над остальными, которое заключается в том, что водопад можно разбить на несколько частей и каждую часть обрабатывать конкретным подходом - где-то частицами, где-то сетками. Подобная возможность повышает эффективность работы алгоритма за счет ступенчатого подхода к визуализации результата. 


\section{Формализация модели}
Модель водопада будет задаваться такими характеристиками, как:
\begin{itemize} 
    \item высота - высота уступа, с которого будет падать вода. Число типа $float$ в промежутке от 1 до 50 (в метрах);
    \item ширина - по сути, ширина уступа, с которого течет вода. Число типа $float$ в промежутке от 1 до 10 (в метрах);
    \item угол падения - угол, который зависит от уступа, с которого падает водопад. Число типа $float$ от 45 до 90 (в градусах).
\end{itemize}

Также частью сцены будет являться скала, которая будет задаваться геометрическим объектом, который будет постоянной формы. Пользователю будет предоставлена возможность изменить его высоту, ширину и угол наклона одной из его сторон.

\img{60mm}{waterfall_scene.jpg}{Объекты сцены и их параметры}


\section{Выбор алгоритма удаления невидимых ребер и поверхностей}
Для начала выделим несколько свойств, которыми должен обладать выбранный алгоритм, чтобы обеспечить оптимальную работу и реалистичное изображение.

Свойства:
\begin{itemize} 
    \item алгоритм может работать как в объектном пространстве, так и в пространстве изображений;
    \item алгоритм должен быть достаточно быстрым и использовать мало памяти.
\end{itemize}


\subsection{Алгоритм, использующий Z-буфер}
\textit{Суть данного алгоритма} - это использование двух буферов: буфера кадра, в котором хранятся атрибуты каждого пикселя, и Z-буфера, в котором хранятся информация о координате Z для каждого пикселя.

Изначально в Z-буфере находятся минимальные значения Z, а в буфере кадра - информация о пикселях, которые описывают фон. Глубина каждого нового пикселя при подсчете сравнивается со значением, которое уже есть в Z-буфере. В случае, если новый пиксель расположен ближе к наблюдателю, чем предыдущий, то информация о нем заносится в буфер кадра и происходит редактирование Z-буфера.

\textit{Положительной стороной} данного алгоритма является простота реализация, экономия времени, так как нет сортировки.

\textit{Недостатком} - большой объем требуемой памяти, а также сложная реализация эффектов прозрачности.


\subsection{Алгоритм обратной трассировки лучей}
\textit{Суть данного алгоритма} состоит в том, что наблюдатель видит объект с помощью испускаемого света, который согласно законам оптики доходит до наблюдателя некоторым путем. Алгоритм называется “обратной трассировкой” из-за того, что эффективнее выходит отслеживать лучи, которые идут от наблюдателя к объекту.

\textit{Положительным моментом} в этом алгоритме является возможность использования в параллельных вычислительных системах.

\textit{Недостатком} же является то, что требуется большое количество необходимых вычислений.


\subsection{Алгоритм Робертса}
Данный алгоритм решает задачу удаления невидимых ребер и граней только с выпуклыми телами, работает только в объектном пространстве.

У алгоритма три этапа выполнения:

\begin{itemize} 
    \item этап 0 - подготовка данных; 
    \item этап 1 - удаление ребер, экранируемых самим телом; 
    \item этап 2 - удаление невидимых ребер, экранируемых другими телами сцены.
\end{itemize}  

\textit{Положительным моментом} в этом алгоритме является высокая точность вычислений. 

\textit{Недостатком} же является то, что все тела должны быть выпуклыми.

\subsection{Алгоритм художника}
Данный алгоритм работает аналогично тому, как художник рисует картину - то есть сначала рисуются дальние объекты, а затем более близкие. Наиболее распространенная реализация алгоритма - сортировка по глубине, которая заключается в том, что произвольное множество граней сортируется по ближнему расстоянию от наблюдателя, а затем отсортированные грани выводятся на экран в порядке от самой дальней до самой ближней. Данный метод работает лучше для построения сцен, в которых отсутствуют пересекающиеся грани. 

\textit{Положительным моментом} является то, что данный алгоритм требует памяти меньше, чем, например, алгоритм Z-буфера.

\textit{Недостатком} же является то, что у него недостаточно высока реалистичность изображения.

\subsection*{Вывод}
Для удаления невидимых линий выбран алгоритм Z-буфера. Данный алгоритм довольно прост в реализации, работает довольно быстро из-за отсутствия сортировок, а также позволяет добиться хорошей реалистичности, что очень важно при моделировании таких  явлений природы, как водопад.

\section{Выбор метода закрашивания областей}
Существует несколько основных алгоритмов закраски:
\begin{itemize} 
    \item простая закраска;
    \item закраска по Гуро; 
    \item закраска по Фонгу.
\end{itemize}  


\subsection{Простая закраска}

Вся грань закрашивается одним уровнем интенсивности, который высчитывается по \textit{закону Ламберта}.

Этот метод достаточно прост в реализации и совершенно не требователен к ресурсам. Однако плохо подходит для тел вращения и плохо учитывает отраженный свет.


\subsection{Закраска по Гуро}

\textit{Основа закраски по Гуро} – билинейная интерполяция интенсивностей, за счет которой устраняется дискретность изменения интенсивности и создается иллюзия гладкой криволинейной поверхности. Хорошо сочетается с диффузным отражением.


\subsection{Закраска по Фонгу}

\textit{Основа закраски по Фонгу} – билинейная интерполяция векторов нормалей. Достигается лучшая локальная аппроксимация кривизны поверхности. Изображение выходит более реалистичным, зеркальные блики выглядят правдоподобнее, чем в методе закраски по Гуро.

Однако по сравнению с методом Гуро, закраска по Фонгу требует больших вычислительных затрат, так как интерполируются значения векторов нормалей, на основе которых потом вычисляется интенсивность.


\subsection*{Вывод}
Для решения поставленной задачи будет использован метод закраски по Гуро, потому что он отвечает всем требованиям для максимально реалистичной закраски сложных объектов сцены.
 

\section{Существующие программные обеспечения}
\textit{Blender} - бесплатный, кроссплатформенный набор инструментов для работы с 3D с открытым кодом. Он обладает широким функционалом - подходит для моделирования, симуляции, рендеринга, монтажа, записи видео и созадния игр.

Благодаря огромному количеству аддонов, которые создают сами пользователи, можно смоделировать практически все, что угодно. Это относится и к водопаду, который можно пустить по любой поверхности, при этом это будет выглядеть максимально реалистично.

На фотографиях ниже представлен пример такого водопада.
\img{60mm}{waterfall_example1.jpg}{Пример реализации водопада в программе Blender ч.1}
\img{60mm}{waterfall_example2.jpg}{Пример реализации водопада в программе Blender ч.2}
\img{60mm}{waterfall_example3.jpg}{Пример реализации водопада в программе Blender ч.3}
\img{60mm}{waterfall_example4.jpg}{Пример реализации водопада в программе Blender ч.4}
\clearpage

\section*{Вывод}
В данном разделе были формально описаны все методы по визуализации текучей воды, с помощью которых можно получить реализиацию водопада, а также методы удаления невидимых линий и поверхностей и закрашивания областей. В качестве алгоритма визуализации водопада предпочтение отдается методу, который реализует комбинированный подход. В качестве алгоритма удаления невидимых линий и областей - z-буфер, а закраски - по Гуро.